\documentclass{article}
\usepackage{graphicx,amsmath,amssymb,amsfonts,} % Required for inserting images

\setlength{\parindent}{0pt}
\begin{document}


D'après les calculs effectués en tâche C.2 et sachant qu'il n'y a plus que 4 cartes dans le paquet :
Les probabilité pour J2 d'obtenir le couple de cartes (i', j') sachant que J1 a obtenu le couple (i, j) sont:\\

$\mathbb{P}((1,1) | (1,1)) = 0$ \\ \\
$\mathbb{P}((1,1) | (1,2)) = 0$ \\ \\
$\mathbb{P}((1,1) | (1,3)) = 0$ \\ \\
$\mathbb{P}((1,1) | (2,2)) = \frac{2}{4} * \frac{1}{3} = \frac{1}{6}$\\ \\
$\mathbb{P}((1,1) | (2,3)) = \frac{2}{4} * \frac{1}{3} = \frac{1}{6}$\\ \\
$\mathbb{P}((1,1) | (3,3)) = \frac{2}{4} * \frac{1}{3} = \frac{1}{6}$\\ \\

$\mathbb{P}((1,2) | (1,1)) = 0$\\ \\
$\mathbb{P}((1,2) | (1,2)) = \frac{1}{4} * \frac{1}{3} = \frac{1}{12}$\\ \\
$\mathbb{P}((1,2) | (1,3)) = \frac{1}{4} * \frac{2}{3} = \frac{1}{6}$\\ \\
$\mathbb{P}((1,2) | (2,2)) = 0$\\ \\
$\mathbb{P}((1,2) | (2,3)) = \frac{2}{4} * \frac{1}{3} = \frac{1}{6}$\\ \\
$\mathbb{P}((1,2) | (3,3)) = \frac{2}{4} * \frac{2}{3} = \frac{1}{3}$\\ \\

$\mathbb{P}((1,3) | (1,1)) = 0$\\ \\
$\mathbb{P}((1,3) | (1,2)) = \frac{1}{4} * \frac{2}{3} = \frac{1}{6}$\\ \\
$\mathbb{P}((1,3) | (1,3)) = \frac{1}{4} * \frac{1}{3} = \frac{1}{12}$\\ \\
$\mathbb{P}((1,3) | (2,2)) = \frac{2}{4} * \frac{2}{3} = \frac{1}{3}$\\ \\
$\mathbb{P}((1,3) | (2,3)) = \frac{2}{4} * \frac{1}{3} = \frac{1}{6}$\\ \\
$\mathbb{P}((1,3) | (3,3)) = 0$\\ \\


$\mathbb{P}((2,2) | (1,1)) = \frac{2}{4} * \frac{1}{3} = \frac{1}{6}$\\ \\
$\mathbb{P}((2,2) | (1,2)) = 0$\\ \\
$\mathbb{P}((2,2) | (1,3)) = \frac{2}{4} * \frac{1}{3} = \frac{1}{6}$\\ \\
$\mathbb{P}((2,2) | (2,2)) = 0$\\ \\
$\mathbb{P}((2,2) | (2,3)) = 0$\\ \\
$\mathbb{P}((2,2) | (3,3)) = \frac{2}{4} * \frac{1}{3} = \frac{1}{6}$\\ \\

$\mathbb{P}((2,3) | (1,1)) = \frac{2}{4} * \frac{2}{3} = \frac{1}{3}$\\ \\
$\mathbb{P}((2,3) | (1,2)) = \frac{1}{4} * \frac{2}{3} = \frac{1}{6}$\\ \\
$\mathbb{P}((2,3) | (1,3)) = \frac{2}{4} * \frac{1}{3} = \frac{1}{6}$\\ \\
$\mathbb{P}((2,3) | (2,2)) = 0$\\ \\
$\mathbb{P}((2,3) | (2,3)) = \frac{1}{4} * \frac{1}{3} = \frac{1}{12}$\\ \\
$\mathbb{P}((2,3) | (3,3)) = 0$\\ \\

$\mathbb{P}((3,3) | (1,1)) = \frac{2}{4} * \frac{1}{3} = \frac{1}{6}$\\ \\
$\mathbb{P}((3,3) | (1,2)) = \frac{2}{4} * \frac{1}{3} = \frac{1}{6}$\\ \\
$\mathbb{P}((3,3) | (1,3)) = 0$\\ \\
$\mathbb{P}((3,3) | (2,2)) = \frac{2}{4} * \frac{1}{3} = \frac{1}{6}$\\ \\
$\mathbb{P}((3,3) | (2,3)) = 0$\\ \\
$\mathbb{P}((3,3) | (3,3)) = 0$\\ \\


\end{document}
